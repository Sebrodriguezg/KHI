section{Las Ecuaciones del Sistema RRMHD}
    \subsection{Sistema de Leyes de Conservación}
        Presentación de las ecuaciones de conservación de masa $\partial_\mu(\rho u^\mu)=0$ y energía-momento $\partial_\mu T^{\mu\nu}=0$ [7].
    \subsection{El Sistema Aumentado de Maxwell}
        Se presenta el sistema que incluye los pseudo-potenciales $\psi$ y $\phi$ (Multiplicadores de Lagrange Generalizados, GLM) para forzar $\nabla \cdot \mathbf{B}=0$ y la conservación de carga $q$ [8, 9]. Las ecuaciones de evolución incluyen $\partial_t \psi = -\nabla \cdot E + q - \kappa \psi$ y $\partial_t \phi = -\nabla \cdot B - \kappa \phi$ [8, 10].
    \subsection{Ley de Ohm Relativista y Corriente Conductiva}
        Definición de la corriente conductiva $j^c$ [11], donde la resistividad $\eta$ (o conductividad $\sigma$) juega un papel clave en los términos rígidos (stiff terms) del sistema [12].

\section{Análisis de la KHI en el Límite Lineal}
    \subsection{Efecto de la Relatividad y la Magnetización}
        Discusión de cómo el campo magnético, especialmente un componente paralelo, tiende a estabilizar la inestabilidad [4].
    \subsection{Impacto de la Resistividad en la Estabilidad}
        La resistividad introduce difusión magnética, afectando el crecimiento de la inestabilidad [13, 14].

