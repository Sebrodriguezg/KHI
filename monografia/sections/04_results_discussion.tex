\section{Pruebas de Verificación Numérica}
    \subsection{Validación de Choques (Shock Tubes)}
        Resultados de pruebas estándar como el problema de tubo de choque **ST1 o ST2** [27, 28] o la **onda de Alfvén** [29, 30] para demostrar la robustez del código.
    \subsection{Prueba de Lámina de Corriente (Current Sheet)}
        Comparación de la solución numérica con la solución analítica para una lámina de corriente auto-similar, validando el tratamiento de la difusión resistiva [14, 31, 32].

\section{Evolución No Lineal de la KHI en RRMHD}
    \subsection{Morfología de la Inestabilidad: Vórtices y Cizalla}
        Descripción de la formación de estructuras vorticales (vórtices de Kelvin-Helmholtz) impulsadas por la cizalla [4].
    \subsection{Efecto Cuantitativo de la Resistividad}
        Análisis de la evolución de variables (densidad $\rho$, velocidad $u_y$, campo magnético $B_x$) para diferentes valores de conductividad, comparando con el límite ideal ($\eta=0$) [33-36].

\section{Implicaciones Físicas}
    Discusión sobre cómo los resultados obtenidos (disipación magnética o formación de turbulencia) se relacionan con fenómenos astrofísicos como los jets relativistas [37].
