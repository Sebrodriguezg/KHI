\section{Métodos de Captura de Choques de Alta Resolución (HRSC)}
    \subsection{Discretización de Volumen Finito (FV)}
        Uso de esquemas basados en la forma integral de las leyes de conservación para garantizar la conservación de las cantidades del modelo [15, 16].

\section{Tratamiento de la Rigidez del Sistema RRMHD}
    \subsection{Esquemas IMEX Runge-Kutta}
        El sistema RRMHD es rígido (stiff) debido a la alta conductividad, requiriendo un tratamiento especial para los términos de relajación [10, 12, 17]. Los métodos **IMEX-RK** (Implícitos-Explícitos) se utilizan para aplicar discretización implícita a los términos rígidos (como la evolución de $\mathbf{E}$) y explícita a los no rígidos [18, 19].
    \subsection{Estrategias de Limpieza de Divergencia}
        Uso de la técnica de **limpieza de divergencia hiperbólica (GLM)**, donde los pseudo-potenciales se integran analíticamente para la parte rígida [20, 21].

\section{Construcción del Esquema de Flujo}
    \subsection{Técnicas de Reconstrucción de Alta Orden}
        Mención a métodos de reconstrucción (e.g., PPM, WENO) para lograr alta precisión espacial y evitar oscilaciones espurias cerca de discontinuidades [20, 22, 23].
    \subsection{Solucionadores de Riemann}
        Uso de solucionadores aproximados como **HLL, Rusanov o HLLC** [24, 25]. El solucionador HLLC es muy robusto y se utiliza para capturar la discontinuidad de contacto [25, 26].

\section{Configuración Inicial (Set-up) de la KHI}
    Especificación de la Ecuación de Estado (EoS) (e.g., gas ideal con $\Gamma=4/3$) y las condiciones iniciales de velocidad $v_x$ y campo magnético $\mathbf{B}$ [8].
