%====================================================================
% PREAMBLE
%====================================================================
\documentclass[11pt, english]{book}

% --- Core Packages ---
\usepackage[T1]{fontenc}                % Use modern font encodings
\usepackage{lmodern}                    % Use the Latin Modern font
\usepackage[utf8]{inputenc}             % For UTF-8 input
\usepackage[spanish]{babel}             % Language-specific typesetting
\usepackage{microtype}                  % Improves justification and typography
\usepackage{csquotes}


% --- Mathematics Packages ---
\usepackage{amsmath}
\usepackage{mathtools}
\usepackage{amssymb}
\usepackage{amsfonts}
\usepackage{bm}           % For bold math symbols, e.g., \bm{v}
\usepackage{derivative}   % For derivatives notation

% --- Graphics and Referencing ---
\usepackage{graphicx}                   % For including images
\usepackage[capitalize]{cleveref}       % For smart cross-referencing (use \cref{})

% --- Bibliography ---
\usepackage[
	backend=biber,
	style=numeric-comp,
	sorting=none
]{biblatex}
\addbibresource{references.bib}         % Link the bibliography file

% --- Theorem-like Environments ---
\newtheorem{theorem}{Theorem}[chapter]
\newtheorem{definition}[theorem]{Definition} % Uses the same counter as theorem
\newtheorem{lemma}[theorem]{Lemma}
\newtheorem{proposition}[theorem]{Proposition}


%====================================================================
% DOCUMENT METADATA
%====================================================================
\title{Dirac Equation}
\author{Bryan Martinez, Sebastían Rodriguez, Serigio Miranda \\ \ \\
  \textit{Programa Académico de Física} \\
  \textit{Facultad de Ciencias Matemáticas y Naturales} \\
  \textit{Universidad Distrital Francisco José de Caldas}
}
\date{\today}


%====================================================================
% DOCUMENT BODY
%====================================================================
\begin{document}

% --- Front Matter ---
\frontmatter
\maketitle
\tableofcontents

% --- Main Matter ---
\mainmatter




% =============================================
% PARTES PRELIMINARES
% =============================================
\chapter*{Agradecimientos}
\chapter*{Resumen}
    Breve descripción del problema (KHI) y el marco (RRMHD), el método (FV, HRSC, IMEX-RK) y resultados principales (efecto de la resistividad).
\chapter*{Lista de Figuras}
\chapter*{Lista de Tablas}
\chapter*{Glosario de Símbolos y Acrónimos}
    RRMHD (Magnetohidrodinámica Resistiva Relativista), KHI (Inestabilidad de Kelvin-Helmholtz), GLM (Multiplicadores de Lagrange Generalizados), EoS (Ecuación de Estado), IMEX-RK, etc.


% =============================================
% CAPÍTULOS CENTRALES
% =============================================

\chapter{Introducción y Contexto Astrofísico}
\label{ch:intro}
\section{Contexto de Plasmas Relativistas y Astrofísicos}
    \subsection{Relevancia de la RRMHD en Chorros y AGN}
        La dinámica de fluidos magnetizados en astrofísica es crucial [1]. La RRMHD es necesaria para modelar fenómenos de plasma en entornos altamente energéticos [2, 3].
    \subsection{La Inestabilidad de Kelvin-Helmholtz (KHI)}
        La KHI surge de la diferencia de velocidad (shear) entre capas de fluidos o plasmas [4]. Su estudio es fundamental en la dinámica de fluidos elementales [5].

\section{Transición de RMHD a RRMHD}
    Justificación para incluir la resistividad ($\eta$ o $1/\sigma$) para modelar plasmas no ideales y efectos de difusión [6].

\section{Objetivos y Estructura de la Monografía}



\chapter{Marco Teórico: Fundamentos de RRMHD}
\label{ch:rrmhd_theory}
section{Las Ecuaciones del Sistema RRMHD}
    \subsection{Sistema de Leyes de Conservación}
        Presentación de las ecuaciones de conservación de masa $\partial_\mu(\rho u^\mu)=0$ y energía-momento $\partial_\mu T^{\mu\nu}=0$ [7].
    \subsection{El Sistema Aumentado de Maxwell}
        Se presenta el sistema que incluye los pseudo-potenciales $\psi$ y $\phi$ (Multiplicadores de Lagrange Generalizados, GLM) para forzar $\nabla \cdot \mathbf{B}=0$ y la conservación de carga $q$ [8, 9]. Las ecuaciones de evolución incluyen $\partial_t \psi = -\nabla \cdot E + q - \kappa \psi$ y $\partial_t \phi = -\nabla \cdot B - \kappa \phi$ [8, 10].
    \subsection{Ley de Ohm Relativista y Corriente Conductiva}
        Definición de la corriente conductiva $j^c$ [11], donde la resistividad $\eta$ (o conductividad $\sigma$) juega un papel clave en los términos rígidos (stiff terms) del sistema [12].

\section{Análisis de la KHI en el Límite Lineal}
    \subsection{Efecto de la Relatividad y la Magnetización}
        Discusión de cómo el campo magnético, especialmente un componente paralelo, tiende a estabilizar la inestabilidad [4].
    \subsection{Impacto de la Resistividad en la Estabilidad}
        La resistividad introduce difusión magnética, afectando el crecimiento de la inestabilidad [13, 14].


\chapter{Metodología Numérica y Configuración del Experimento}
\label{ch:numerical_methods}
\section{Métodos de Captura de Choques de Alta Resolución (HRSC)}
    \subsection{Discretización de Volumen Finito (FV)}
        Uso de esquemas basados en la forma integral de las leyes de conservación para garantizar la conservación de las cantidades del modelo [15, 16].

\section{Tratamiento de la Rigidez del Sistema RRMHD}
    \subsection{Esquemas IMEX Runge-Kutta}
        El sistema RRMHD es rígido (stiff) debido a la alta conductividad, requiriendo un tratamiento especial para los términos de relajación [10, 12, 17]. Los métodos **IMEX-RK** (Implícitos-Explícitos) se utilizan para aplicar discretización implícita a los términos rígidos (como la evolución de $\mathbf{E}$) y explícita a los no rígidos [18, 19].
    \subsection{Estrategias de Limpieza de Divergencia}
        Uso de la técnica de **limpieza de divergencia hiperbólica (GLM)**, donde los pseudo-potenciales se integran analíticamente para la parte rígida [20, 21].

\section{Construcción del Esquema de Flujo}
    \subsection{Técnicas de Reconstrucción de Alta Orden}
        Mención a métodos de reconstrucción (e.g., PPM, WENO) para lograr alta precisión espacial y evitar oscilaciones espurias cerca de discontinuidades [20, 22, 23].
    \subsection{Solucionadores de Riemann}
        Uso de solucionadores aproximados como **HLL, Rusanov o HLLC** [24, 25]. El solucionador HLLC es muy robusto y se utiliza para capturar la discontinuidad de contacto [25, 26].

\section{Configuración Inicial (Set-up) de la KHI}
    Especificación de la Ecuación de Estado (EoS) (e.g., gas ideal con $\Gamma=4/3$) y las condiciones iniciales de velocidad $v_x$ y campo magnético $\mathbf{B}$ [8].


\chapter{Resultados de la Simulación y Análisis de la KHI}
\label{ch:results_discussion}
\section{Pruebas de Verificación Numérica}
    \subsection{Validación de Choques (Shock Tubes)}
        Resultados de pruebas estándar como el problema de tubo de choque **ST1 o ST2** [27, 28] o la **onda de Alfvén** [29, 30] para demostrar la robustez del código.
    \subsection{Prueba de Lámina de Corriente (Current Sheet)}
        Comparación de la solución numérica con la solución analítica para una lámina de corriente auto-similar, validando el tratamiento de la difusión resistiva [14, 31, 32].

\section{Evolución No Lineal de la KHI en RRMHD}
    \subsection{Morfología de la Inestabilidad: Vórtices y Cizalla}
        Descripción de la formación de estructuras vorticales (vórtices de Kelvin-Helmholtz) impulsadas por la cizalla [4].
    \subsection{Efecto Cuantitativo de la Resistividad}
        Análisis de la evolución de variables (densidad $\rho$, velocidad $u_y$, campo magnético $B_x$) para diferentes valores de conductividad, comparando con el límite ideal ($\eta=0$) [33-36].

\section{Implicaciones Físicas}
    Discusión sobre cómo los resultados obtenidos (disipación magnética o formación de turbulencia) se relacionan con fenómenos astrofísicos como los jets relativistas [37].


\chapter{Conclusiones y Perspectivas Futuras}
\label{ch:conclusions}
\section{Resumen de Contribuciones}
    Recapitulación de la implementación robusta del esquema numérico (IMEX-RK, FV, HRSC) y los hallazgos sobre la amortiguación/estabilización de la KHI por la resistividad.
\section{Limitaciones del Modelo y la Simulación}
    Mencionar la simplicidad de la EoS ($\Gamma=4/3$) o la restricción a 2D (si aplica).
\section{Trabajo Futuro}
    Sugerencias para futuras investigaciones, como la extensión a geometrías más complejas (3D), la inclusión de la gravedad (GRMHD), o el uso de esquemas de reconstrucción de orden aún mayor (WENO7) [20].


% =============================================
% APÉNDICES
% =============================================
\appendix
\chapter{Apéndices}
\section{Detalles Vectoriales en Coordenadas Cartesianas}
    Listado de identidades vectoriales clave [38, 39].

\section{Construcción de Flujos Numéricos y Solvers}
    Detalles sobre la estructura de la discretización semi-discreta [40]. Mención a los "bloques numéricos" como los limitadores de pendiente (slope limiters) utilizados en la reconstrucción (e.g., minmod) [41, 42].

\section{Parámetros de Simulación Detallados}
    Tabla de constantes físicas ($\Gamma$), parámetros de régimen (e.g., Mach number, $\mu_p, \mu_t$ [8]), y parámetros numéricos (CCFL, resolución, $\sigma$).













% --- Back Matter ---
% Contains appendices, bibliography, index.
\backmatter

\printbibliography % This command prints the bibliography

\end{document}
