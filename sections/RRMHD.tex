\section{Magnetohidrodinámica Resistiva Relativista (RRMHD)}
La \textbf{magnetohidrodinámica resistiva relativista} (RRMHD) describe la evolución de un plasma magnetizado cuando los efectos relativistas y resistivos no pueden ser despreciados. 
En este marco, el plasma se trata como un \textit{fluido continuo cargado}, cuya dinámica está acoplada a las ecuaciones de Maxwell.  
El sistema se formula usualmente en el marco del \textbf{fluido perfecto relativista}, extendido para incluir una conductividad finita $\sigma$, que introduce un término resistivo en la ley de Ohm generalizada:
\[
J^\mu = \rho_e u^\mu + \sigma F^{\mu\nu}u_\nu,
\]
donde $J^\mu$ es la corriente de cuatro-vector, $u^\mu$ el cuadrivelocidad y $F^{\mu\nu}$ el tensor electromagnético.

La RRMHD combina las ecuaciones de conservación de masa, energía y momento, junto con las ecuaciones de Maxwell aumentadas mediante pseudo-potenciales $(\psi, \phi)$ para controlar las divergencias de los campos eléctricos y magnéticos, siguiendo el esquema propuesto por Dedner et al. (2002).

---

\section{Deducción y Representaciones de las Ecuaciones de la RRMHD}
Las ecuaciones fundamentales de la RRMHD se derivan a partir de:
\begin{enumerate}
    \item La \textbf{conservación del tensor energía–momento} $T^{\mu\nu}_{\ \ ;\nu}=0$.
    \item La \textbf{conservación de la carga} $J^\mu_{\ ;\mu}=0$.
    \item Las \textbf{ecuaciones de Maxwell relativistas}.
\end{enumerate}

De estas relaciones se obtienen tres formas equivalentes de representación:
\begin{enumerate}
    \item \textit{Forma covariante}, útil para formulaciones analíticas y simulaciones generales relativistas.
    \item \textit{Forma conservativa}, empleada en métodos numéricos de volúmenes finitos.
    \item \textit{Forma aumentada}, que introduce las variables $\psi$ y $\phi$ para mantener las condiciones $\nabla \cdot \mathbf{B}=0$ y $\nabla \cdot \mathbf{E}=q$.
\end{enumerate}

---

\section{La Inestabilidad de Kelvin–Helmholtz}
La \textbf{inestabilidad de Kelvin–Helmholtz} (KHI) surge en la interfaz entre dos capas de fluido (o plasma) con velocidades tangenciales diferentes.  
Pequeñas perturbaciones en la interfaz pueden crecer exponencialmente, generando vórtices y estructuras turbulentas.  
En el contexto magnetohidrodinámico, el campo magnético puede suprimir o modificar el crecimiento de esta inestabilidad dependiendo de su orientación y magnitud.

---

\section{Contexto Astrofísico e Implicaciones}
La KHI se observa en diversos escenarios astrofísicos:
\begin{itemize}
    \item Límites de chorros relativistas (jets de AGNs y microcuásares).
    \item Fronteras entre vientos estelares y medio interestelar.
    \item Discos de acreción y regiones de cizalla en supernovas.
\end{itemize}
Su estudio es clave para comprender la mezcla de plasmas, la generación de turbulencia y el transporte de energía en medios magnetizados relativistas.

---

\section{Estado Actual del Problema}
En la actualidad, se ha logrado una comprensión cualitativa del efecto estabilizador del campo magnético y la resistividad.  
Sin embargo, aún existen retos abiertos:
\begin{itemize}
    \item Determinar el papel preciso de la resistividad anómala.
    \item Extender simulaciones 2D a 3D conservando estabilidad numérica.
    \item Acoplar RRMHD con radiación para describir jets altamente luminosos.
\end{itemize}

---

\section{Método Numérico}
El tratamiento numérico se basa en un esquema de \textbf{volúmenes finitos relativista}, utilizando reconstrucción de variables conservadas y un método de Riemann aproximado.  
La integración temporal se realiza mediante esquemas de tipo Runge–Kutta de segundo orden.  
El control de las condiciones $\nabla \cdot \mathbf{B}=0$ se maneja mediante el sistema aumentado de Dedner, introduciendo términos de amortiguamiento proporcional a $\kappa$:
\[
\partial_t \phi = -\nabla\cdot \mathbf{B} - \kappa \phi.
\]

---

\section{Simulaciones Numéricas y Resultados}
Se realizaron simulaciones 2D del desarrollo de la KHI variando la conductividad eléctrica $\sigma$.  
El set-up inicial considera una perturbación sinusoidal en $v_y$:
\[
v_y = A_0 v_{sh} \sin(2\pi x)\, e^{-\frac{(y\pm 0.5)^2}{\alpha^2}},
\]
y una cizalla tangencial en $v_x$.  
Se analizaron las magnitudes de presión, densidad y velocidades transversales ($v_y$, $v_z$), observándose que:
\begin{itemize}
    \item La resistividad alta atenúa el crecimiento de las perturbaciones.
    \item La formación de estructuras turbulentas es menos eficiente.
\end{itemize}

---

\section{Próximos Objetivos}
El siguiente paso consiste en extender el estudio a geometrías tridimensionales y explorar el régimen \textit{ultra-relativista}, donde los efectos de colimación y disipación magnética son más pronunciados.  
También se planea analizar la transición de la KHI hacia estados turbulentos completamente desarrollados.

---

\section{Preguntas Potenciales}
\begin{enumerate}
    \item ¿Qué diferencias clave existen entre la MHD clásica y la RRMHD?
    \item ¿Cómo afecta la resistividad al crecimiento de la KHI?
    \item ¿Por qué es importante incluir las ecuaciones aumentadas de Maxwell?
    \item ¿Qué implicaciones tienen los resultados en el contexto de jets relativistas?
    \item ¿Cuál es la principal limitación del método numérico actual?
\end{enumerate}
